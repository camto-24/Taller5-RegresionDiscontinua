% Options for packages loaded elsewhere
\PassOptionsToPackage{unicode}{hyperref}
\PassOptionsToPackage{hyphens}{url}
%
\documentclass[
]{article}
\usepackage{amsmath,amssymb}
\usepackage{iftex}
\ifPDFTeX
  \usepackage[T1]{fontenc}
  \usepackage[utf8]{inputenc}
  \usepackage{textcomp} % provide euro and other symbols
\else % if luatex or xetex
  \usepackage{unicode-math} % this also loads fontspec
  \defaultfontfeatures{Scale=MatchLowercase}
  \defaultfontfeatures[\rmfamily]{Ligatures=TeX,Scale=1}
\fi
\usepackage{lmodern}
\ifPDFTeX\else
  % xetex/luatex font selection
\fi
% Use upquote if available, for straight quotes in verbatim environments
\IfFileExists{upquote.sty}{\usepackage{upquote}}{}
\IfFileExists{microtype.sty}{% use microtype if available
  \usepackage[]{microtype}
  \UseMicrotypeSet[protrusion]{basicmath} % disable protrusion for tt fonts
}{}
\makeatletter
\@ifundefined{KOMAClassName}{% if non-KOMA class
  \IfFileExists{parskip.sty}{%
    \usepackage{parskip}
  }{% else
    \setlength{\parindent}{0pt}
    \setlength{\parskip}{6pt plus 2pt minus 1pt}}
}{% if KOMA class
  \KOMAoptions{parskip=half}}
\makeatother
\usepackage{xcolor}
\usepackage[top=1.5cm, bottom=1.5cm, left=2cm, right=2cm]{geometry}
\usepackage{graphicx}
\makeatletter
\def\maxwidth{\ifdim\Gin@nat@width>\linewidth\linewidth\else\Gin@nat@width\fi}
\def\maxheight{\ifdim\Gin@nat@height>\textheight\textheight\else\Gin@nat@height\fi}
\makeatother
% Scale images if necessary, so that they will not overflow the page
% margins by default, and it is still possible to overwrite the defaults
% using explicit options in \includegraphics[width, height, ...]{}
\setkeys{Gin}{width=\maxwidth,height=\maxheight,keepaspectratio}
% Set default figure placement to htbp
\makeatletter
\def\fps@figure{htbp}
\makeatother
\setlength{\emergencystretch}{3em} % prevent overfull lines
\providecommand{\tightlist}{%
  \setlength{\itemsep}{0pt}\setlength{\parskip}{0pt}}
\setcounter{secnumdepth}{-\maxdimen} % remove section numbering
\usepackage{adjustbox}
\ifLuaTeX
  \usepackage{selnolig}  % disable illegal ligatures
\fi
\usepackage{bookmark}
\IfFileExists{xurl.sty}{\usepackage{xurl}}{} % add URL line breaks if available
\urlstyle{same}
\hypersetup{
  pdftitle={Taller 5 - Regresión Discontinua},
  pdfauthor={Maria Camila Caraballo, Laura Sarif Rivera},
  hidelinks,
  pdfcreator={LaTeX via pandoc}}

\title{Taller 5 - Regresión Discontinua}
\author{Maria Camila Caraballo, Laura Sarif Rivera}
\date{}

\begin{document}
\maketitle

\subsection{Taller Regresión
Discontinua}\label{taller-regresiuxf3n-discontinua}

En su artículo ``Islamic Rule and the Empowerment of the Poor and
Pious'', Meyersson (2014) investiga si la llegada al poder por parte del
Partido Islámico tiene algún efecto sobre el empoderamiento de las
mujeres en Turquía. Para esto, implementa la metodología de Regresión
Discontinua, explotando información de: (1) elecciones locales de
alcalde en Turquía del año 1994 y (2) mujeres con educación secundaria
completa en el año 2000. Concretamente, estima por MCO la siguiente
ecuación

\[y_i=α+βm_i+f(x_i )+ε_i\]

donde \(Y_i\) es la proporción de mujeres entre 15 y 20 años con
educación secundaria completa en el año 2000, \(x_i\) es el margen de
votos con el que ganó o perdió el candidato del partido islámico,
\(f(.)\) es un polinomio de grado n de la variable \(x_i\), \(m_i\) es
una dicótoma que toma el valor de uno si \(x_i > 0\), es decir, si el
alcalde que llegó al poder en 1994 era del partido Islámico y
\(\epsilon_i\) es el término del error. La ecuación es estimada en un
vecindario alrededor del corte, el cual, en este caso, es cero.

\subsubsection{Preparación de los
datos}\label{preparaciuxf3n-de-los-datos}

Para esto se deberá eliminar aleatoriamente el 5\% de las observaciones
y usar la base restante. La semilla que deben usar para que sus
resultados sean replicables es su código de estudiante.

\subsubsection{1. Análisis
introductorio}\label{anuxe1lisis-introductorio}

\textbf{¿Por qué el autor usa la metodología de Regresión Discontinua
para identificar los efectos de interés? ¿Cuál es la intuición detrás?
¿Cuál es el supuesto de identificación?}

Erik Meyersson (2014) en su trabajo \emph{Islamic Rule and the
Empowerment of the Poor and Pious} busca identificar el efecto causal de
que un municipio sea gobernado por un alcalde islámico sobre el
empoderamiento de las mujeres, medido a través de la proporción de
jóvenes entre 15 y 20 años con educación secundaria completa. El desafío
metodológico radica en que la elección de un partido islámico no es
aleatoria, ya que los municipios que votan por estos partidos pueden
diferir en características culturales, religiosas o socioeconómicas que
también influyen en los resultados educativos. Si se compararan
directamente municipios con y sin alcalde islámico, las estimaciones
resultarían sesgadas por esas diferencias estructurales.

La metodología de regresión discontinua ofrece una solución porque
permite aislar la variación que puede considerarse casi aleatoria en
torno al umbral de victoria electoral. Cuando un candidato gana o pierde
por un margen muy pequeño, el resultado depende más del azar o de
factores impredecibles que de diferencias estructurales entre los
municipios. En ese punto de corte, ganar o perder una elección es casi
como producto del azar, lo que genera un entorno cuasi experimental en
el que las unidades a ambos lados del umbral son muy parecidas.
Conceptualmente, los municipios que se ubican justo por encima o por
debajo del margen de victoria del partido islámico tienen el mismo
perfil social, económico y cultural, por lo que cualquier salto en los
resultados educativos observado exactamente en el punto de corte puede
interpretarse como consecuencia directa de haber tenido un alcalde
islámico.

El supuesto fundamental de identificación es el de continuidad, consiste
en que en ausencia del tratamiento el resultado esperado variaría de
manera continua con el margen de victoria electoral. Esto implica que no
existen saltos sistemáticos en el resultado alrededor del punto de corte
que no estén causados por la victoria del partido islamista. Es decir,
el supuesto sostiene que cerca del umbral los municipios son similares
en todo aspecto salvo en la variable de tratamiento, lo que permite
interpretar la discontinuidad observada en los resultados como un efecto
causal del triunfo electoral del partido.

\newpage

\subsubsection{2. Para cada género, presenten en una tabla los
resultados de estimar las siguientes
especificacione}\label{para-cada-guxe9nero-presenten-en-una-tabla-los-resultados-de-estimar-las-siguientes-especificacione}

\textbf{a. Ecuación principal, para toda la muestra, sin incluir
controles}

{[}1{]} ``\,''\\
{[}2{]} ``\textbackslash begin\{table\}{[}ht!{]}
\textbackslash centering''\\
{[}3{]} '' \textbackslash caption\{Efecto de alcaldes islámicos sobre
educación secundaria\} ''\\
{[}4{]} '' \textbackslash label\{tab:rd\_mujeres\_hombres\} ''\\
{[}5{]}
``\textbackslash begin\{tabular\}\{@\textbackslash\textbackslash extracolsep\{5pt\}lcc\}''\\
{[}6{]}
``\textbackslash\textbackslash{[}-1.8ex{]}\textbackslash hline''\\
{[}7{]} ``\textbackslash hline
\textbackslash\textbackslash{[}-1.8ex{]}''\\
{[}8{]} '' \&
\textbackslash multicolumn\{2\}\{c\}\{\textbackslash textit\{Dependent
variable:\}\} \textbackslash\textbackslash{} ''\\
{[}9{]} ``\textbackslash cline\{2-3\}''\\
{[}10{]} '' \& Mujeres \& Hombres \textbackslash\textbackslash{} ''\\
{[}11{]} ``\textbackslash\textbackslash{[}-1.8ex{]} \& (1) \&
(2)\textbackslash\textbackslash{}''\\
{[}12{]} ``\textbackslash hline
\textbackslash\textbackslash{[}-1.8ex{]}''\\
{[}13{]} '' Alcalde islámico (Dᵢ) \& \$-\$0.00001 (0.0001) \& 0.008
(0.007) \textbackslash\textbackslash{} ''\\
{[}14{]} '' xi \& \$-\$0.0002 (0.0002) \& 0.011 (0.014)
\textbackslash\textbackslash{} ''\\
{[}15{]} '' di:xi \& \$-\(0.002\)\^{}\{*\textbf{\}\$ (0.001) \&
\$-\(0.092\)\^{}\{}\}\$ (0.041) \textbackslash\textbackslash{} ''\\
{[}16{]} '' Constant \& 0.002\(^{***}\) (0.0001) \& 0.196\(^{***}\)
(0.005) \textbackslash\textbackslash{} ''\\
{[}17{]} '' \textbackslash hline
\textbackslash\textbackslash{[}-1.8ex{]} ''\\
{[}18{]} ``Observations \& 2,498 \& 2,498
\textbackslash\textbackslash{}''\\
{[}19{]} ``R\(^{2}\) \& 0.014 \& 0.002
\textbackslash\textbackslash{}''\\
{[}20{]} ``Adjusted R\(^{2}\) \& 0.013 \& 0.001
\textbackslash\textbackslash{}''\\
{[}21{]} ``Residual Std. Error (df = 2494) \& 0.001 \& 0.077
\textbackslash\textbackslash{}''\\
{[}22{]} ``F Statistic (df = 3; 2494) \& 12.095\(^{***}\) \& 1.661
\textbackslash\textbackslash{}''\\
{[}23{]} ``\textbackslash hline''\\
{[}24{]} ``\textbackslash hline
\textbackslash\textbackslash{[}-1.8ex{]}''\\
{[}25{]} ``\textbackslash textit\{Note:\} \&
\textbackslash multicolumn\{2\}\{r\}\{\(^{*}\)p\$\textless\$0.1;
\(^{**}\)p\$\textless\$0.05; \(^{***}\)p\$\textless\$0.01\}
\textbackslash\textbackslash{}'' {[}26{]} '' \&
\textbackslash multicolumn\{2\}\{r\}\{Errores estándar agrupados por
provincia.\} \textbackslash\textbackslash{} ''\\
{[}27{]} ``\textbackslash end\{tabular\}''\\
{[}28{]} ``\textbackslash end\{table\}''\\
{[}29{]} ``\,''\\
{[}30{]} ``\textbackslash begin\{table\}{[}ht!{]}
\textbackslash centering''\\
{[}31{]} '' \textbackslash caption\{Efecto de alcaldes islámicos sobre
educación secundaria\} ''\\
{[}32{]} '' \textbackslash label\{tab:rd\_mujeres\_hombres\} ''\\
{[}33{]}
``\textbackslash begin\{tabular\}\{@\textbackslash\textbackslash extracolsep\{5pt\}
c\}''\\
{[}34{]}
``\textbackslash\textbackslash{[}-1.8ex{]}\textbackslash hline''\\
{[}35{]} ``\textbackslash hline
\textbackslash\textbackslash{[}-1.8ex{]}''\\
{[}36{]} ``TRUE \textbackslash\textbackslash{}''\\
{[}37{]} ``\textbackslash hline
\textbackslash\textbackslash{[}-1.8ex{]}''\\
{[}38{]} ``\textbackslash multicolumn\{1\}\{l\}\{Errores estándar
agrupados por provincia.\} \textbackslash\textbackslash{}''\\
{[}39{]} ``\textbackslash end\{tabular\}''\\
{[}40{]} ``\textbackslash end\{table\}''

\newpage

\textbf{b. Ecuación principal, para toda la muestra, con controles.}

{[}1{]} ``\,''\\
{[}2{]} ``\textbackslash begin\{table\}{[}ht!{]}
\textbackslash centering''\\
{[}3{]} '' \textbackslash caption\{Efecto de alcaldes islámicos sobre
educación secundaria\} ''\\
{[}4{]} '' \textbackslash label\{tab:rd\_controles\_mujeres\_hombres\}
''\\
{[}5{]}
``\textbackslash begin\{tabular\}\{@\textbackslash\textbackslash extracolsep\{5pt\}lcc\}''\\
{[}6{]}
``\textbackslash\textbackslash{[}-1.8ex{]}\textbackslash hline''\\
{[}7{]} ``\textbackslash hline
\textbackslash\textbackslash{[}-1.8ex{]}''\\
{[}8{]} '' \&
\textbackslash multicolumn\{2\}\{c\}\{\textbackslash textit\{Dependent
variable:\}\} \textbackslash\textbackslash{} ''\\
{[}9{]} ``\textbackslash cline\{2-3\}''\\
{[}10{]} '' \& Mujeres \& Hombres \textbackslash\textbackslash{} ''\\
{[}11{]} ``\textbackslash\textbackslash{[}-1.8ex{]} \& (1) \&
(2)\textbackslash\textbackslash{}''\\
{[}12{]} ``\textbackslash hline
\textbackslash\textbackslash{[}-1.8ex{]}''\\
{[}13{]} '' Alcalde islámico (Dᵢ) \& 0.0001\(^{**}\) (0.0001) \& 0.005
(0.006) \textbackslash\textbackslash{} ''\\
{[}14{]} '' Número de partidos (1994) \&
\$-\(0.0004\)\^{}\{\textbf{\emph{\}\$ (0.0001) \& 0.015 (0.014)
\textbackslash\textbackslash{} ''\\
{[}15{]} '' Población (log, 1994) \& \$-\(0.001\)\^{}\{}}\}\$ (0.0004)
\& \$-\(0.069\)\^{}\{\textbf{\}\$ (0.033) \textbackslash\textbackslash{}
''\\
{[}16{]} '' Proporción 0--19 años \& \$-\(0.00003\)\^{}\{}\}\$ (0.00002)
\& \$-\(0.010\)\^{}\{\textbf{\emph{\}\$ (0.002)
\textbackslash\textbackslash{} ''\\
{[}17{]} '' Proporción 60+ años \& 0.0001\(^{**}\) (0.00003) \& 0.005
(0.003) \textbackslash\textbackslash{} ''\\
{[}18{]} '' Razón de sexo \& \$-\(0.0001\)\^{}\{}}\}\$ (0.00001) \&
\$-\(0.005\)\^{}\{\textbf{\emph{\}\$ (0.001)
\textbackslash\textbackslash{} ''\\
{[}19{]} '' Tamaño promedio del hogar \& \$-\(0.0001\)\^{}\{}}\}\$
(0.00001) \& \$-\(0.005\)\^{}\{\textbf{\emph{\}\$ (0.001)
\textbackslash\textbackslash{} ''\\
{[}20{]} '' Categoría: Merkezi \& \$-\(0.00000\)\^{}\{}}\}\$ (0.00000)
\& \$-\(0.0004\)\^{}\{***\}\$ (0.0001) \textbackslash\textbackslash{}
''\\
{[}21{]} '' Categoría: Merkezp \& 0.00005\(^{**}\) (0.00002) \&
0.006\(^{***}\) (0.002) \textbackslash\textbackslash{} ''\\
{[}22{]} '' Categoría: Sub-Büyük \& 0.001\(^{***}\) (0.0001) \&
0.047\(^{***}\) (0.005) \textbackslash\textbackslash{} ''\\
{[}23{]} '' Categoría: Büyük \& 0.0005\(^{***}\) (0.0001) \&
0.052\(^{***}\) (0.010) \textbackslash\textbackslash{} ''\\
{[}24{]} '' subbuyuk \& 0.0003 (0.0002) \& 0.037\(^{***}\) (0.014)
\textbackslash\textbackslash{} ''\\
{[}25{]} '' buyuk \& 0.0004\(^{*}\) (0.0002) \& 0.079\(^{***}\) (0.020)
\textbackslash\textbackslash{} ''\\
{[}26{]} '' Constant \& 0.005\(^{***}\) (0.0004) \& 0.472\(^{***}\)
(0.039) \textbackslash\textbackslash{} ''\\
{[}27{]} '' \textbackslash hline
\textbackslash\textbackslash{[}-1.8ex{]} ''\\
{[}28{]} ``Observations \& 2,498 \& 2,498
\textbackslash\textbackslash{}''\\
{[}29{]} ``R\(^{2}\) \& 0.438 \& 0.192
\textbackslash\textbackslash{}''\\
{[}30{]} ``Adjusted R\(^{2}\) \& 0.436 \& 0.187
\textbackslash\textbackslash{}''\\
{[}31{]} ``Residual Std. Error (df = 2484) \& 0.001 \& 0.070
\textbackslash\textbackslash{}''\\
{[}32{]} ``F Statistic (df = 13; 2484) \& 149.220\(^{***}\) \&
45.323\(^{***}\) \textbackslash\textbackslash{}''\\
{[}33{]} ``\textbackslash hline''\\
{[}34{]} ``\textbackslash hline
\textbackslash\textbackslash{[}-1.8ex{]}''\\
{[}35{]} ``\textbackslash textit\{Note:\} \&
\textbackslash multicolumn\{2\}\{r\}\{\(^{*}\)p\$\textless\$0.1;
\(^{**}\)p\$\textless\$0.05; \(^{***}\)p\$\textless\$0.01\}
\textbackslash\textbackslash{}''\\
{[}36{]} '' \& \textbackslash multicolumn\{2\}\{r\}\{Errores estándar
agrupados por provincia. Todos los modelos incluyen efectos fijos por
provincia.\} \textbackslash\textbackslash{} '' {[}37{]}
``\textbackslash end\{tabular\}''\\
{[}38{]} ``\textbackslash end\{table\}''\\
{[}39{]} ``\,''\\
{[}40{]} ``\textbackslash begin\{table\}{[}ht!{]}
\textbackslash centering''\\
{[}41{]} '' \textbackslash caption\{Efecto de alcaldes islámicos sobre
educación secundaria\} ''\\
{[}42{]} '' \textbackslash label\{tab:rd\_controles\_mujeres\_hombres\}
''\\
{[}43{]}
``\textbackslash begin\{tabular\}\{@\textbackslash\textbackslash extracolsep\{5pt\}
c\}''\\
{[}44{]}
``\textbackslash\textbackslash{[}-1.8ex{]}\textbackslash hline''\\
{[}45{]} ``\textbackslash hline
\textbackslash\textbackslash{[}-1.8ex{]}''\\
{[}46{]} ``TRUE \textbackslash\textbackslash{}''\\
{[}47{]} ``\textbackslash hline
\textbackslash\textbackslash{[}-1.8ex{]}''\\
{[}48{]} ``\textbackslash multicolumn\{1\}\{l\}\{Errores estándar
agrupados por provincia. Todos los modelos incluyen efectos fijos por
provincia.\} \textbackslash\textbackslash{}''\\
{[}49{]} ``\textbackslash end\{tabular\}''\\
{[}50{]} ``\textbackslash end\{table\}''

\begin{table}[ht!] \centering 
  \caption{Efecto de alcaldes islámicos sobre educación secundaria} 
  \label{tab:rd_controles_mujeres_hombres} 
\begin{tabular}{@{\extracolsep{5pt}}lcc} 
\\[-1.8ex]\hline 
\hline \\[-1.8ex] 
 & \multicolumn{2}{c}{\textit{Dependent variable:}} \\ 
\cline{2-3} 
 & Mujeres & Hombres \\ 
\\[-1.8ex] & (1) & (2)\\ 
\hline \\[-1.8ex] 
 Alcalde islámico (Dᵢ) & 0.0001$^{**}$ (0.0001) & 0.005 (0.006) \\ 
  Número de partidos (1994) & $-$0.0004$^{***}$ (0.0001) & 0.015 (0.014) \\ 
  Población (log, 1994) & $-$0.001$^{***}$ (0.0004) & $-$0.069$^{**}$ (0.033) \\ 
  Proporción 0–19 años & $-$0.00003$^{**}$ (0.00002) & $-$0.010$^{***}$ (0.002) \\ 
  Proporción 60+ años & 0.0001$^{**}$ (0.00003) & 0.005 (0.003) \\ 
  Razón de sexo & $-$0.0001$^{***}$ (0.00001) & $-$0.005$^{***}$ (0.001) \\ 
  Tamaño promedio del hogar & $-$0.0001$^{***}$ (0.00001) & $-$0.005$^{***}$ (0.001) \\ 
  Categoría: Merkezi & $-$0.00000$^{***}$ (0.00000) & $-$0.0004$^{***}$ (0.0001) \\ 
  Categoría: Merkezp & 0.00005$^{**}$ (0.00002) & 0.006$^{***}$ (0.002) \\ 
  Categoría: Sub-Büyük & 0.001$^{***}$ (0.0001) & 0.047$^{***}$ (0.005) \\ 
  Categoría: Büyük & 0.0005$^{***}$ (0.0001) & 0.052$^{***}$ (0.010) \\ 
  subbuyuk & 0.0003 (0.0002) & 0.037$^{***}$ (0.014) \\ 
  buyuk & 0.0004$^{*}$ (0.0002) & 0.079$^{***}$ (0.020) \\ 
  Constant & 0.005$^{***}$ (0.0004) & 0.472$^{***}$ (0.039) \\ 
 \hline \\[-1.8ex] 
Observations & 2,498 & 2,498 \\ 
R$^{2}$ & 0.438 & 0.192 \\ 
Adjusted R$^{2}$ & 0.436 & 0.187 \\ 
Residual Std. Error (df = 2484) & 0.001 & 0.070 \\ 
F Statistic (df = 13; 2484) & 149.220$^{***}$ & 45.323$^{***}$ \\ 
\hline 
\hline \\[-1.8ex] 
\textit{Note:}  & \multicolumn{2}{r}{$^{*}$p$<$0.1; $^{**}$p$<$0.05; $^{***}$p$<$0.01} \\ 
 & \multicolumn{2}{r}{Errores estándar agrupados por provincia. Todos los modelos incluyen efectos fijos por provincia.} \\ 
\end{tabular} 
\end{table}

\begin{table}[ht!] \centering 
  \caption{Efecto de alcaldes islámicos sobre educación secundaria} 
  \label{tab:rd_controles_mujeres_hombres} 
\begin{tabular}{@{\extracolsep{5pt}} c} 
\\[-1.8ex]\hline 
\hline \\[-1.8ex] 
TRUE \\ 
\hline \\[-1.8ex] 
\multicolumn{1}{l}{Errores estándar agrupados por provincia. Todos los modelos incluyen efectos fijos por provincia.} \\ 
\end{tabular} 
\end{table}

\newpage

Donde h=0.24 es el ancho de banda óptimo estimado por los autores. Para
las especificaciones con controles, supongan que:

\[f(x_i )= γx _i+δx_i×m_i\]

Es decir, un polinomio de grado uno con pendiente distinta a cada lado
del corte. Para las especificaciones sin controles, no incluyan ningún
polinomio. Todas las especificaciones deben usar errores estándar
clúster a nivel de provincia.

\textbf{c.~Ecuación principal, para la submuestra a h unidades alrededor
del corte, con controles.}

{[}1{]} ``\,''\\
{[}2{]} ``\textbackslash begin\{table\}{[}ht!{]}
\textbackslash centering''\\
{[}3{]} '' \textbackslash caption\{Regresión Discontinua con controles
(submuestra ĥ alrededor del corte)\} ''\\
{[}4{]} ''
\textbackslash label\{tab:rd\_submuestra\_h\_mujeres\_hombres\} ''\\
{[}5{]}
``\textbackslash begin\{tabular\}\{@\textbackslash\textbackslash extracolsep\{5pt\}lcc\}''\\
{[}6{]}
``\textbackslash\textbackslash{[}-1.8ex{]}\textbackslash hline''\\
{[}7{]} ``\textbackslash hline
\textbackslash\textbackslash{[}-1.8ex{]}''\\
{[}8{]} '' \&
\textbackslash multicolumn\{2\}\{c\}\{\textbackslash textit\{Dependent
variable:\}\} \textbackslash\textbackslash{} ''\\
{[}9{]} ``\textbackslash cline\{2-3\}''\\
{[}10{]} '' \& Mujeres \& Hombres \textbackslash\textbackslash{} ''\\
{[}11{]} ``\textbackslash\textbackslash{[}-1.8ex{]} \& (1) \&
(2)\textbackslash\textbackslash{}''\\
{[}12{]} ``\textbackslash hline
\textbackslash\textbackslash{[}-1.8ex{]}''\\
{[}13{]} '' Alcalde islámico (Dᵢ) \& 0.0003\(^{***}\) (0.0001) \& 0.010
(0.008) \textbackslash\textbackslash{} ''\\
{[}14{]} '' Número de partidos (1994) \& \$-\(0.001\)\^{}\{\textbf{\}\$
(0.0004) \& 0.004 (0.040) \textbackslash\textbackslash{} ''\\
{[}15{]} '' Población (log, 1994) \& \$-\(0.002\)\^{}\{}\}\$ (0.001) \&
\$-\$0.084 (0.067) \textbackslash\textbackslash{} ''\\
{[}16{]} '' Proporción 0--19 años \& \$-\$0.00002 (0.00002) \&
\$-\(0.009\)\^{}\{\textbf{\emph{\}\$ (0.002)
\textbackslash\textbackslash{} ''\\
{[}17{]} '' Proporción 60+ años \& 0.0001\(^{**}\) (0.00004) \& 0.004
(0.004) \textbackslash\textbackslash{} ''\\
{[}18{]} '' Razón de sexo \& \$-\(0.0001\)\^{}\{}}\}\$ (0.00001) \&
\$-\(0.005\)\^{}\{\textbf{\emph{\}\$ (0.0005)
\textbackslash\textbackslash{} ''\\
{[}19{]} '' Tamaño promedio del hogar \& \$-\(0.00004\)\^{}\{}}\}\$
(0.00002) \& \$-\(0.005\)\^{}\{\textbf{\emph{\}\$ (0.001)
\textbackslash\textbackslash{} ''\\
{[}20{]} '' Categoría: Merkezi \& 0.00000 (0.00000) \&
\$-\(0.0001 (0.0003) \\\\ "
[21] "  Categoría: Merkezp & 0.00004\)\^{}\{}}\}\$ (0.00001) \&
0.005\(^{***}\) (0.001) \textbackslash\textbackslash{} ''\\
{[}22{]} '' Categoría: Sub-Büyük \& 0.001\(^{***}\) (0.0001) \&
0.052\(^{***}\) (0.006) \textbackslash\textbackslash{} ''\\
{[}23{]} '' Categoría: Büyük \& 0.001\(^{***}\) (0.0001) \&
0.047\(^{***}\) (0.011) \textbackslash\textbackslash{} ''\\
{[}24{]} '' subbuyuk \& 0.0002 (0.0002) \& 0.027 (0.018)
\textbackslash\textbackslash{} ''\\
{[}25{]} '' buyuk \& 0.0004\(^{*}\) (0.0002) \& 0.071\(^{***}\) (0.022)
\textbackslash\textbackslash{} ''\\
{[}26{]} '' Constant \& 0.004\(^{***}\) (0.001) \& 0.445\(^{***}\)
(0.065) \textbackslash\textbackslash{} ''\\
{[}27{]} '' \textbackslash hline
\textbackslash\textbackslash{[}-1.8ex{]} ''\\
{[}28{]} ``Observations \& 968 \& 968 \textbackslash\textbackslash{}''\\
{[}29{]} ``R\(^{2}\) \& 0.548 \& 0.268
\textbackslash\textbackslash{}''\\
{[}30{]} ``Adjusted R\(^{2}\) \& 0.542 \& 0.258
\textbackslash\textbackslash{}''\\
{[}31{]} ``Residual Std. Error (df = 954) \& 0.001 \& 0.065
\textbackslash\textbackslash{}''\\
{[}32{]} ``F Statistic (df = 13; 954) \& 88.888\(^{***}\) \&
26.828\(^{***}\) \textbackslash\textbackslash{}''\\
{[}33{]} ``\textbackslash hline''\\
{[}34{]} ``\textbackslash hline
\textbackslash\textbackslash{[}-1.8ex{]}''\\
{[}35{]} ``\textbackslash textit\{Note:\} \&
\textbackslash multicolumn\{2\}\{r\}\{\(^{*}\)p\$\textless\$0.1;
\(^{**}\)p\$\textless\$0.05; \(^{***}\)p\$\textless\$0.01\}
\textbackslash\textbackslash{}''\\
{[}36{]} '' \& \textbackslash multicolumn\{2\}\{r\}\{Errores estándar
agrupados por provincia. Submuestra: ±ĥ unidades alrededor del punto de
corte. Incluye controles y efectos fijos por provincia.\}
\textbackslash\textbackslash{} '' {[}37{]}
``\textbackslash end\{tabular\}''\\
{[}38{]} ``\textbackslash end\{table\}''\\
{[}39{]} ``\,''\\
{[}40{]} ``\textbackslash begin\{table\}{[}ht!{]}
\textbackslash centering''\\
{[}41{]} '' \textbackslash caption\{Regresión Discontinua con controles
(submuestra ĥ alrededor del corte)\} ''\\
{[}42{]} ''
\textbackslash label\{tab:rd\_submuestra\_h\_mujeres\_hombres\} ''\\
{[}43{]}
``\textbackslash begin\{tabular\}\{@\textbackslash\textbackslash extracolsep\{5pt\}
c\}''\\
{[}44{]}
``\textbackslash\textbackslash{[}-1.8ex{]}\textbackslash hline''\\
{[}45{]} ``\textbackslash hline
\textbackslash\textbackslash{[}-1.8ex{]}''\\
{[}46{]} ``TRUE \textbackslash\textbackslash{}''\\
{[}47{]} ``\textbackslash hline
\textbackslash\textbackslash{[}-1.8ex{]}''\\
{[}48{]} ``\textbackslash multicolumn\{1\}\{l\}\{Errores estándar
agrupados por provincia. Submuestra: ±ĥ unidades alrededor del punto de
corte. Incluye controles y efectos fijos por provincia.\}
\textbackslash\textbackslash{}''\\
{[}49{]} ``\textbackslash end\{tabular\}''\\
{[}50{]} ``\textbackslash end\{table\}''

\begin{table}[ht!] \centering 
  \caption{Regresión Discontinua con controles (submuestra ĥ alrededor del corte)} 
  \label{tab:rd_submuestra_h_mujeres_hombres} 
\begin{tabular}{@{\extracolsep{5pt}}lcc} 
\\[-1.8ex]\hline 
\hline \\[-1.8ex] 
 & \multicolumn{2}{c}{\textit{Dependent variable:}} \\ 
\cline{2-3} 
 & Mujeres & Hombres \\ 
\\[-1.8ex] & (1) & (2)\\ 
\hline \\[-1.8ex] 
 Alcalde islámico (Dᵢ) & 0.0003$^{***}$ (0.0001) & 0.010 (0.008) \\ 
  Número de partidos (1994) & $-$0.001$^{**}$ (0.0004) & 0.004 (0.040) \\ 
  Población (log, 1994) & $-$0.002$^{**}$ (0.001) & $-$0.084 (0.067) \\ 
  Proporción 0–19 años & $-$0.00002 (0.00002) & $-$0.009$^{***}$ (0.002) \\ 
  Proporción 60+ años & 0.0001$^{**}$ (0.00004) & 0.004 (0.004) \\ 
  Razón de sexo & $-$0.0001$^{***}$ (0.00001) & $-$0.005$^{***}$ (0.0005) \\ 
  Tamaño promedio del hogar & $-$0.00004$^{***}$ (0.00002) & $-$0.005$^{***}$ (0.001) \\ 
  Categoría: Merkezi & 0.00000 (0.00000) & $-$0.0001 (0.0003) \\ 
  Categoría: Merkezp & 0.00004$^{***}$ (0.00001) & 0.005$^{***}$ (0.001) \\ 
  Categoría: Sub-Büyük & 0.001$^{***}$ (0.0001) & 0.052$^{***}$ (0.006) \\ 
  Categoría: Büyük & 0.001$^{***}$ (0.0001) & 0.047$^{***}$ (0.011) \\ 
  subbuyuk & 0.0002 (0.0002) & 0.027 (0.018) \\ 
  buyuk & 0.0004$^{*}$ (0.0002) & 0.071$^{***}$ (0.022) \\ 
  Constant & 0.004$^{***}$ (0.001) & 0.445$^{***}$ (0.065) \\ 
 \hline \\[-1.8ex] 
Observations & 968 & 968 \\ 
R$^{2}$ & 0.548 & 0.268 \\ 
Adjusted R$^{2}$ & 0.542 & 0.258 \\ 
Residual Std. Error (df = 954) & 0.001 & 0.065 \\ 
F Statistic (df = 13; 954) & 88.888$^{***}$ & 26.828$^{***}$ \\ 
\hline 
\hline \\[-1.8ex] 
\textit{Note:}  & \multicolumn{2}{r}{$^{*}$p$<$0.1; $^{**}$p$<$0.05; $^{***}$p$<$0.01} \\ 
 & \multicolumn{2}{r}{Errores estándar agrupados por provincia. Submuestra: ±ĥ unidades alrededor del punto de corte. Incluye controles y efectos fijos por provincia.} \\ 
\end{tabular} 
\end{table}

\begin{table}[ht!] \centering 
  \caption{Regresión Discontinua con controles (submuestra ĥ alrededor del corte)} 
  \label{tab:rd_submuestra_h_mujeres_hombres} 
\begin{tabular}{@{\extracolsep{5pt}} c} 
\\[-1.8ex]\hline 
\hline \\[-1.8ex] 
TRUE \\ 
\hline \\[-1.8ex] 
\multicolumn{1}{l}{Errores estándar agrupados por provincia. Submuestra: ±ĥ unidades alrededor del punto de corte. Incluye controles y efectos fijos por provincia.} \\ 
\end{tabular} 
\end{table}

\newpage

\textbf{d.~Ecuación principal, para la submuestra a h/2 unidades
alrededor del corte, con controles}

{[}1{]} ``\,''\\
{[}2{]} ``\textbackslash begin\{table\}{[}ht!{]}
\textbackslash centering''\\
{[}3{]} '' \textbackslash caption\{Regresión Discontinua con controles y
±ĥ/)\} ''\\
{[}4{]} ''
\textbackslash label\{tab:rd\_submuestra\_h2\_mujeres\_hombres\} ''\\
{[}5{]}
``\textbackslash begin\{tabular\}\{@\textbackslash\textbackslash extracolsep\{5pt\}lcc\}''\\
{[}6{]}
``\textbackslash\textbackslash{[}-1.8ex{]}\textbackslash hline''\\
{[}7{]} ``\textbackslash hline
\textbackslash\textbackslash{[}-1.8ex{]}''\\
{[}8{]} '' \&
\textbackslash multicolumn\{2\}\{c\}\{\textbackslash textit\{Dependent
variable:\}\} \textbackslash\textbackslash{} ''\\
{[}9{]} ``\textbackslash cline\{2-3\}''\\
{[}10{]} '' \& Mujeres \& Hombres \textbackslash\textbackslash{} ''\\
{[}11{]} ``\textbackslash\textbackslash{[}-1.8ex{]} \& (1) \&
(2)\textbackslash\textbackslash{}''\\
{[}12{]} ``\textbackslash hline
\textbackslash\textbackslash{[}-1.8ex{]}''\\
{[}13{]} '' Alcalde islámico (Dᵢ) \& 0.0003\(^{**}\) (0.0001) \&
0.019\(^{*}\) (0.011) \textbackslash\textbackslash{} ''\\
{[}14{]} '' Número de partidos (1994) \& \$-\(0.002\)\^{}\{\emph{\}\$
(0.001) \& \$-\$0.005 (0.080) \textbackslash\textbackslash{} ''\\
{[}15{]} '' Población (log, 1994) \& 0.0001 (0.001) \& \$-\$0.235
(0.155) \textbackslash\textbackslash{} ''\\
{[}16{]} '' Proporción 0--19 años \& \$-\$0.00002 (0.00002) \&
\$-\(0.008\)\^{}\{}\textbf{\}\$ (0.002) \textbackslash\textbackslash{}
''\\
{[}17{]} '' Proporción 60+ años \& 0.0001 (0.00005) \& 0.006 (0.005)
\textbackslash\textbackslash{} ''\\
{[}18{]} '' Razón de sexo \& \$-\(0.0001\)\^{}\{}\emph{\}\$ (0.00001) \&
\$-\(0.005\)\^{}\{}**\}\$ (0.001) \textbackslash\textbackslash{} ''\\
{[}19{]} '' Tamaño promedio del hogar \& \$-\(0.00004\)\^{}\{\emph{\}\$
(0.00002) \& \$-\(0.004\)\^{}\{\textbf{\}\$ (0.001)
\textbackslash\textbackslash{} ''\\
{[}20{]} '' Categoría: Merkezi \& 0.00000 (0.00000) \&
\$-\(0.0001 (0.0003) \\\\ "
[21] "  Categoría: Merkezp & 0.00004\)\^{}\{}}\}\$ (0.00002) \&
0.004\(^{***}\) (0.002) \textbackslash\textbackslash{} ''\\
{[}22{]} '' Categoría: Sub-Büyük \& 0.001\(^{***}\) (0.0001) \&
0.057\(^{***}\) (0.008) \textbackslash\textbackslash{} ''\\
{[}23{]} '' Categoría: Büyük \& 0.001\(^{***}\) (0.0002) \&
0.038\(^{**}\) (0.016) \textbackslash\textbackslash{} ''\\
{[}24{]} '' subbuyuk \& 0.0002 (0.0002) \& 0.007 (0.013)
\textbackslash\textbackslash{} ''\\
{[}25{]} '' buyuk \& 0.0004 (0.0003) \& 0.034 (0.023)
\textbackslash\textbackslash{} ''\\
{[}26{]} '' Constant \& 0.004\(^{***}\) (0.001) \& 0.378\(^{***}\)
(0.077) \textbackslash\textbackslash{} ''\\
{[}27{]} '' \textbackslash hline
\textbackslash\textbackslash{[}-1.8ex{]} ''\\
{[}28{]} ``Observations \& 560 \& 560 \textbackslash\textbackslash{}''\\
{[}29{]} ``R\(^{2}\) \& 0.533 \& 0.283
\textbackslash\textbackslash{}''\\
{[}30{]} ``Adjusted R\(^{2}\) \& 0.522 \& 0.265
\textbackslash\textbackslash{}''\\
{[}31{]} ``Residual Std. Error (df = 546) \& 0.001 \& 0.064
\textbackslash\textbackslash{}''\\
{[}32{]} ``F Statistic (df = 13; 546) \& 47.980\(^{***}\) \&
16.539\(^{***}\) \textbackslash\textbackslash{}''\\
{[}33{]} ``\textbackslash hline''\\
{[}34{]} ``\textbackslash hline
\textbackslash\textbackslash{[}-1.8ex{]}''\\
{[}35{]} ``\textbackslash textit\{Note:\} \&
\textbackslash multicolumn\{2\}\{r\}\{\(^{*}\)p\$\textless\$0.1;
\(^{**}\)p\$\textless\$0.05; \(^{***}\)p\$\textless\$0.01\}
\textbackslash\textbackslash{}''\\
{[}36{]} '' \& \textbackslash multicolumn\{2\}\{r\}\{Errores estándar
agrupados por provincia. Submuestra: ±ĥ/2 unidades alrededor del punto
de corte. Incluye controles y efectos fijos por provincia.\}
\textbackslash\textbackslash{} '' {[}37{]}
``\textbackslash end\{tabular\}''\\
{[}38{]} ``\textbackslash end\{table\}''\\
{[}39{]} ``\,''\\
{[}40{]} ``\textbackslash begin\{table\}{[}ht!{]}
\textbackslash centering''\\
{[}41{]} '' \textbackslash caption\{Regresión Discontinua con controles
y ±ĥ/)\} ''\\
{[}42{]} ''
\textbackslash label\{tab:rd\_submuestra\_h2\_mujeres\_hombres\} ''\\
{[}43{]}
``\textbackslash begin\{tabular\}\{@\textbackslash\textbackslash extracolsep\{5pt\}
c\}''\\
{[}44{]}
``\textbackslash\textbackslash{[}-1.8ex{]}\textbackslash hline''\\
{[}45{]} ``\textbackslash hline
\textbackslash\textbackslash{[}-1.8ex{]}''\\
{[}46{]} ``TRUE \textbackslash\textbackslash{}''\\
{[}47{]} ``\textbackslash hline
\textbackslash\textbackslash{[}-1.8ex{]}''\\
{[}48{]} ``\textbackslash multicolumn\{1\}\{l\}\{Errores estándar
agrupados por provincia. Submuestra: ±ĥ/2 unidades alrededor del punto
de corte. Incluye controles y efectos fijos por provincia.\}
\textbackslash\textbackslash{}''\\
{[}49{]} ``\textbackslash end\{tabular\}''\\
{[}50{]} ``\textbackslash end\{table\}''

\newpage

\subsubsection{3. A partir de los resultados encontrados en el anterior
punto,
respondan:}\label{a-partir-de-los-resultados-encontrados-en-el-anterior-punto-respondan}

\textbf{a. ¿Por qué cambian los coeficientes entre especificaciones?}

Los coeficientes cambian entre especificaciones porque cada modelo
incorpora distintos supuestos y grados de control sobre posibles
problemas de sesgo. En la primera estimación, que no incluye variables
de control, el modelo capta únicamente la discontinuidad simple en torno
al umbral de victoria electoral. Al introducir controles socioeconómicos
y demográficos en las siguientes especificaciones, el modelo mejora la
estimación del efecto del alcalde islámico de otras características
municipales que también pueden influir en la educación. Asimismo, al
restringir la muestra a bandas más estrechas alrededor del punto de
corte, se reduce la heterogeneidad entre observaciones, lo que mejora la
validez causal pero puede aumentar la varianza de la estimación. Por
estas razones, los coeficientes cambian ligeramente entre
especificaciones, reflejando cómo la inclusión de controles y la
reducción del ancho de banda afectan la precisión y el sesgo potencial
del estimador

\textbf{b. ¿Cuál parece ser el impacto de la llegada al poder del
Partido Islámico para las mujeres? ¿Parece ser este impacto robusto a
las especificaciones?}

En cuanto al efecto de la llegada al poder del partido islámico sobre la
educación de las mujeres, los resultados muestran que este impacto es
positivo y estadísticamente significativo en las especificaciones que
integran controles y restringen la muestra alrededor del umbral. En
particular, en los modelos de regresión discontinua con controles, tanto
en la muestra con ancho de banda óptimo como en la reducida a la mitad,
el coeficiente asociado a tener un alcalde islámico es positivo y
significativo al 1\% o 5\%. Esto sugiere que la presencia de un gobierno
municipal islámico se asocia con una mayor proporción de mujeres jóvenes
que completan la educación secundaria.

Finalmente, el hecho de que los coeficientes se mantengan positivos y
significativos para las mujeres en las distintas especificaciones indica
que el resultado es robusto. A pesar de los cambios en la inclusión de
controles o en el tamaño de la muestra, la dirección y magnitud del
efecto son consistentes, mientras que para los hombres los coeficientes
no son estadísticamente significativos en la mayoría de los casos. Esto
refuerza la interpretación de que el impacto de los alcaldes islámicos
es específico al grupo femenino y que los resultados no dependen de una
especificación particular del modelo, sino que se sostienen bajo
diferentes estrategias de estimación.

\subsubsection{4. Finalmente, presenten evidencia a favor (o en contra)
del supuesto de identificación, para
esto:}\label{finalmente-presenten-evidencia-a-favor-o-en-contra-del-supuesto-de-identificaciuxf3n-para-esto}

\textbf{a. Presenten en una gráfica de la distribución kernel o el
histograma de la variable de asignación}\(x_i\) \textbf{¿Parece haber
manipulación?}

\includegraphics{Taller5_RD_CamilaCaraballo_LauraRivera_files/figure-latex/unnamed-chunk-8-1.pdf}

El gráfico no indica la presencia de manipulación en el umbral de
asignación. El supuesto de no manipulación en el diseño de regresión
discontinua requiere que la densidad de observaciones en la variable de
running sea continua en el punto de corte (que en este caso se marca en
el punto 0). La densidad que se observa es suave y no presenta un
quiebre, un salto o una caída abrupta en el umbral. Esta continuidad es
una evidencia fundamental que sugiere que la probabilidad de ganar o
perder por un margen pequeño es practicamente la misma, lo cual valida
la asignación del tratamiento como cuasi-aleatoria.

\newpage

\textbf{b. En una tabla presenten los resultados de estimar la ecuación
de interés, sin controles, pero incluyendo} \(f(x_i)\) \textbf{tomando
como variables dependientes: i) la elección de un alcalde del partido
Islámico en 1984 (i89) y ii) el logaritmo de la población en 1994
(lpop1994). Para esto, usen únicamente la muestra de elecciones
alrededor del ancho de banda óptimo. Dados sus resultados, ¿parece haber
continuidad en estas variables?}

\begin{table}[ht!] \centering 
  \caption{Prueba de continuidad en variables específicas} 
  \label{tab:continuidad_predeterminadas} 
\begin{tabular}{@{\extracolsep{5pt}}lcc} 
\\[-1.8ex]\hline 
\hline \\[-1.8ex] 
 & \multicolumn{2}{c}{\textit{Dependent variable:}} \\ 
\cline{2-3} 
\\[-1.8ex] & Alcalde islámico en 1984 & Log población 1994 \\ 
\\[-1.8ex] & (1) & (2)\\ 
\hline \\[-1.8ex] 
 Ganó alcalde islámico & $-$0.005 (0.048) & 0.181 (0.183) \\ 
  Margen electoral & 0.375$^{***}$ (0.117) & $-$0.947 (0.903) \\ 
  Interacción margen x ganador & 0.064 (0.452) & 2.004 (1.904) \\ 
  Constant & 0.090$^{***}$ (0.020) & 8.058$^{***}$ (0.153) \\ 
 \hline \\[-1.8ex] 
Observations & 718 & 968 \\ 
R$^{2}$ & 0.030 & 0.004 \\ 
Adjusted R$^{2}$ & 0.026 & 0.001 \\ 
Residual Std. Error & 0.244 (df = 714) & 1.508 (df = 964) \\ 
F Statistic & 7.254$^{***}$ (df = 3; 714) & 1.254 (df = 3; 964) \\ 
\hline 
\hline \\[-1.8ex] 
\textit{Note:}  & \multicolumn{2}{r}{$^{*}$p$<$0.1; $^{**}$p$<$0.05; $^{***}$p$<$0.01} \\ 
\end{tabular} 
\end{table}

\begin{table}[ht!] \centering 
  \caption{Prueba de continuidad en variables específicas} 
  \label{tab:continuidad_predeterminadas} 
\begin{tabular}{@{\extracolsep{5pt}} c} 
\\[-1.8ex]\hline 
\hline \\[-1.8ex] 
TRUE \\ 
\hline \\[-1.8ex] 
\end{tabular} 
\end{table}

\newpage

\textbf{c.~Dados sus resultados en los incisos a y b, ¿es plausible el
supuesto de identificación? Expliquen por qué.}

Los resultados indican una continuidad en las covariables analizadas.
Esta prueba permite confirmar que variables que no deberían ser
afectadas por el resultado electoral de 1994 no muestran un salto
significativo en el punto de corte. La continuidad se verifica al
observar que el coeficiente estimado, que captura la discontinuidad en
el punto de corte, es estadísticamente indistinguible de cero para las
variables analizadas. Específicamente, el valor P amplio para el Alcalde
Islámico en 1989 (0.781) y para el Logaritmo de la Población en 1994
(0.866) demuestra que no existe una diferencia significativa en el
historial político o el tamaño poblacional entre los municipios que
apenas ganaron y los que apenas perdieron la elección de 1994. Esta
ausencia de un salto brusco en las covariables antes del tratamiento
refuerza la credibilidad de que la asignación del tratamiento es
aleatoria en el umbral, lo cual es el supuesto de identificación central
del diseño de regresión discontinua.

\end{document}
